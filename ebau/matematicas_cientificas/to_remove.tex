\documentclass{article}
\usepackage[utf8]{inputenc}
\usepackage[spanish]{babel}
\usepackage{amsmath, amssymb}

\title{Introducción a la Teoría de Galois}
\author{}
\date{}

\begin{document}

\maketitle

\section*{Álgebra Abstracta y Teoría de Galois}

La \textbf{Teoría de Galois} es una rama fundamental del álgebra abstracta que estudia la relación entre las raíces de los polinomios y las simetrías presentes en dichas raíces. Desarrollada por el matemático francés Évariste Galois en el siglo XIX, esta teoría proporciona criterios para determinar si una ecuación polinómica puede resolverse mediante radicales, es decir, utilizando operaciones aritméticas básicas y extracción de raíces. :contentReference[oaicite:0]{index=0}

Un concepto central en esta teoría es el de \textbf{grupo de Galois}, que consiste en el conjunto de todas las permutaciones de las raíces de un polinomio que conservan las relaciones algebraicas entre ellas. La estructura de este grupo permite comprender la solubilidad de la ecuación original. :contentReference[oaicite:1]{index=1}

La Teoría de Galois ha tenido un impacto significativo en la resolución de problemas clásicos de la matemática, como la imposibilidad de resolver por radicales ecuaciones polinómicas de grado cinco o superior, y ha sentado las bases para el desarrollo del álgebra moderna y la teoría de grupos. :contentReference[oaicite:2]{index=2}

My biggest steps, 
\end{document}
